\documentclass[article,a4paper,12pt,twoside]{memoir}
%% We require ``twoside'' in order for the left/right pages to have different headers
%% But this then requires us to solve some problems with margin pars not showing up
%% That

%% For double spacing
\DisemulatePackage{setspace}
\usepackage{setspace}

%% For turning off hyphenation globally
% \usepackage[none]{hyphenat}
\usepackage{hyphenat}

\usepackage{amsmath}
\usepackage[MnSymbol]{mathspec}
\setmathsfont(Digits,Latin,Greek)[Numbers={Lining,Proportional}]{Linux Libertine O}
\setminwhitespace % not sure if this helps
\setmathrm{Linux Biolinum O}

\usepackage{xunicode}
%\defaultfontfeatures{Mapping=tex-text}
\setmainfont[Mapping= tex-text,
  SmallCapsFont={Linux Libertine O},
  SmallCapsFeatures= {Color=FF2400, RawFeature={+smcp,+hlig,+dlig}}]{Linux Libertine O}

\usepackage{metalogo}

% This gets around having too few math fonts in a cheap way.
% However you can't just copy and paste mathcal stuff
% Not sure why!
\newfontfamily\calligraphicfont{Linux Biolinum Shadow O}
\newcommand\calP{\text{\calligraphicfont P}}
\newcommand\calU{\text{\calligraphicfont U}}
\newcommand\calV{\text{\calligraphicfont V}}
\newcommand\calF{\text{\calligraphicfont F}}

% Fun colors and ligatures - maybe too fun
%% \setmainfont[Mapping= tex-text,
%%     SmallCapsFont={Linux Libertine O},
%%     SmallCapsFeatures= {Color=000000, RawFeature={+smcp,+hlig,+dlig}},
%%     BoldFont={Linux Libertine O Bold},
%%     BoldFeatures={Color = 000000,SmallCapsFont={Linux Libertine Capitals O Bold},%
%%       SmallCapsFeatures = { Color=70CF14,   RawFeature={+smcp,+hlig,+dlig}} },
%%     ItalicFont={Linux Libertine O Italic},
%%     ItalicFeatures={Color = 000000, RawFeature={+liga,+hlig,+dlig}, %
%%       SmallCapsFont={Linux Libertine Capitals O Italic}, %
%%       SmallCapsFeatures = {Color=000000,RawFeature={+smcp,+hlig,+dlig}}},
%%     BoldItalicFont={Linux Libertine O},
%%     BoldItalicFeatures={ Color = 000000, %
%%      SmallCapsFont={Linux Libertine Capitals O Bold Italic},  %
%%      SmallCapsFeatures = { Color=000000,RawFeature={+smcp,+hlig,+dlig}}} ]{Linux Libertine O}


\usepackage{placeins}
\usepackage[super]{natbib}
\usepackage[perpage,symbol*]{footmisc}

\usepackage{fancyvrb}

%% Use ``ex'' to make the spacing depend on current font, cool!
\DefineFNsymbols*{safestar}{{$*$}{\textdagger}{\textdaggerdbl}{\textsection}{\textparagraph}{\textbardbl}{$*$\hspace{-.3ex}$*$}{\textdagger\textdagger}{\textdaggerdbl\textdaggerdbl}{\textsection\textsection}{\textparagraph\textparagraph}{\textbardbl\textbardbl}}%
\setfnsymbol{safestar}

\usepackage[normalem]{ulem}
\usepackage{tikz}

\usepackage{wrapfig}

 \usepackage{xspace}
 \makeatletter
  \def\stex{\raisebox{-.5ex}S\kern-.5ex\TeX\xspace}
 \makeatother
 \def\sTeX{\stex}

\newcommand*\myendnote[1]{%
  % The ``side'' where this goes is set above.
  % Contrary to page 250 of the memoir manual, it seems the default is ``inner''.

  %%% but Alex and Peter don't seem to think it is that important to have any endnotes
  %%% in any case, so I'm taking this out for now
  %\sidepar{\endnote{#1}}
}

\newcommand*\circled[1]{%
  \tikz[baseline=(C.base)]\node[draw,circle,inner sep=0.5pt](C) {#1};\!
}

\newcommand\marktransition{\begin{center}◆\end{center}}

\usepackage{indentfirst}
\usepackage{marvosym}

\usepackage{rotating}

%%%%
%% Thesis guidelines:
%%  your full name and degrees
%%  the title
%%  the degree for which it is offered
%%  the discipline or disciplines to which it pertains
%%  the date of submission
%%%%%

%%  inside margin           40 mm
%%  top and outside margins 15 mm
%%  bottom margin           20 mm

%% With the ``twoside'' setting for the document class, these
%% accomplish the above, giving 5mm extra on the outside margin
%% and 5mm extra for the top margin
\setlrmarginsandblock{40mm}{20mm}{*}
\setulmarginsandblock{20mm}{20mm}{*}

%% Another version that's a little more comfortable
%% but not usable for the standards of the Open University
% \setlrmarginsandblock{23mm}{23mm}{*}
% \setulmarginsandblock{23mm}{28mm}{*}

% Parameters here are: { separation }{ width }{ push }
\setmarginnotes{14pt}{51pt}{\onelineskip}
\sideparmargin{inner}

\setheadfoot{2\onelineskip}{2\onelineskip}
\setheaderspaces{*}{5mm}{*}

\chapterstyle{plain}

%%%
\checkandfixthelayout
\setsecnumformat{\hspace{-.82in}{\HUGE \thesection}\hspace{.23in}}

%% ``promote'' sections to format them like chapters:
%% Tip c/o egreg at http://bit.ly/memoir-promote-sections
\renewcommand{\thesection}{\arabic{section}}
\makeatletter
\let\l@section\l@chapter
\makeatother

%\renewcommand*\thesection{\arabic{section}}
%\renewcommand*\thesection{\arabic{chapter}}

% following a nice tip from http://bit.ly/SZcZO9
\strictpagecheck
\makeatletter
\newcommand*\ifthispageodd{%
  \checkoddpage
  \ifoddpage
    \expandafter\@firstoftwo
  \else
    \expandafter\@secondoftwo
  \fi
}
\makeatother

\usepackage{makeidx}
\makeindex

\setsecindent{1cm}
\let\footruleskip\relax % for compatibility of memoir and fancyhdr
\let\proportional\ttfamily % for compatibility of memoir and blindtext
\let\sc\scshape
\let\bf\bfseries
\let\it\itfamily
\let\tt\ttfamily
\let\sl\slshape
\let\sf\sffamily

% This is the header implementation, it took me a little while to figure out why
% it works like this!
\usepackage{fancyhdr}
\pagestyle{fancy}
\fancyhead[LE]{\slshape \thesection.~ \nouppercase{\leftmark}}
\fancyhead[RO]{\slshape \nouppercase{\rightmark}}
\fancyhead[RE]{}
\fancyhead[LO]{}
\cfoot{\thepage}
\renewcommand{\headrulewidth}{0pt}

\renewcommand{\sectionmark}[1]{\markboth{#1}{}}
\renewcommand{\subsectionmark}[1]{\markright{#1}}
%\DisemulatePackage{moreverb} % compatibility with pdftricks

\parindent = 1.4em

%%% Legacy math stuff
%\usepackage{makeidx}  % allows for indexgeneration
%\usepackage{graphicx}
\newenvironment{qq}
  {\begin{quote}\noindent}%
  {\end{quote}}

\newenvironment{cc}
  {\begin{quote}\noindent}%
  {\end{quote}}

\newcommand{\openset}{\mathrel{{\mathchoice{\rlap{$\subset$}{\;\scriptscriptstyle\cdot}}%
   {\rlap{$\subset$}{\;\cdot}}%
   {\rlap{$\scriptstyle\subset$}{\;\cdot}}%
   {\rlap{$\scriptscriptstyle\subset$}{\;\cdot}}}}}

%% \newcommand{\sidequote}[1]{
%% \marginpar{{ \begin{rotate}{270}
%% \raisebox{4mm}{\HUGE #1}
%% \end{rotate}
%% }}}

\newcommand{\sidequote}[1]{}

\let\oldindex\index
% Actually we won't use this in ``production''
\renewcommand{\index}[1]{
\oldindex{#1}
\ifthispageodd{
\marginpar{{ \begin{rotate}{270}
\raisebox{8mm}{\hspace{-5mm}\tiny #1}
\end{rotate}
}}}{
\marginpar{{ \begin{rotate}{90}
\raisebox{-10mm}{\hspace{-15mm}\tiny #1}
\end{rotate}
}}}
}
% so let's switch back
\let\index\oldindex

\newcommand{\Rn}{\mathbf{R}^n}
\newcounter{123listcolonstylectr}

\newenvironment{123listcolonstyle}{
\indent \begin{list}{\arabic{123listcolonstylectr}:}{\usecounter{123listcolonstylectr}}}
{\end{list}\setcounter{123listcolonstylectr}{0}}

\def\defn#1{{\footnotesize \indent
\begin{123listcolonstyle}
\setlength{\itemsep}{0em}
\setlength{\topsep}{0em}
\setlength{\parsep}{0em} #1 \end{123listcolonstyle}}}
%%% End of legacy math stuff

%\usepackage{amsmath}
%\usepackage{amsthm}
\usepackage{xcolor}         % colors
\definecolor{scarlet}{HTML}{FF2400}

%% Can use this to make the labels in the back a different font or color

%\makeatletter
%\renewcommand\@biblabel[1]{{\sc #1.}}
%\makeatother

\makeatletter
\renewcommand\@biblabel[1]{#1.}
\makeatother

\PassOptionsToPackage{hyphens}{url}
%% frenchlinks here will make all the links show up in small caps font
\usepackage[frenchlinks, pdfborderstyle={/S/U/W .5},citebordercolor={1 1 1},linkbordercolor={1 1 1},urlbordercolor={1 1 1}]{hyperref}

%% really useful TOC configuration - turn off the far-too-many links!
\hypersetup{linktocpage}

\def\UrlFont{\tt}

\usepackage{endnotes}
\usepackage{hyperendnotes}

% \usepackage{pifont}
\newcommand{\rem}[2]{\begin{center}\framebox[1\textwidth]{\parbox{.92\textwidth}{\hspace{-.037\textwidth} \lefthand \hspace{-.002\textwidth}   \emph{\bf#1}\\ #2}}\end{center}}

\usepackage{framed}

\newenvironment{cframed}[1][blue]
  {\def\FrameCommand{\fboxsep=\FrameSep\fcolorbox{#1}{white}}%
    \MakeFramed {\advance\hsize-\width \FrameRestore}}
  {\endMakeFramed}


\usepackage{enumitem}

\newcommand{\inlinebox}[1]{\medskip\begin{framed}\noindent#1\end{framed}\medskip}

\makeatletter
\newcommand*{\shifttext}[2]{%
  \settowidth{\@tempdima}{#2}%
  \makebox[\@tempdima]{\hspace*{#1}#2}%
}
\makeatother

% \newcommand{\inlineboxb}[1]{\shifttext{.7in}{\framebox{\parbox{.8\textwidth}{#1}}}\\}
\definecolor{shadecolor}{rgb}{.95,0.95,0.95}
\newcommand{\inlineboxb}[1]{\begin{framed}#1\end{framed}}

\newcommand{\PMlinkname}[2]{\href{http://planetmath.org/#2.html}{#1}}

\newenvironment{emptybar}{%
  \def\FrameCommand{\hspace{12pt}}%
  \MakeFramed {\advance\hsize-\width \FrameRestore}}%
 {\endMakeFramed}

\AtBeginDocument{\addtocontents{toc}{\protect\thispagestyle{empty}}}

\setcounter{tocdepth}{4}
\newcounter{example}
\newenvironment{example}[1]{\refstepcounter{example}
\framed
\addcontentsline{toc}{paragraph}{\lefthand~#1}
\hspace{-.048\textwidth} \lefthand \hspace{-.003\textwidth}
{\bf \thechapter.\theexample~\emph{#1}} \emptybar \noindent \hspace{-.5em}}
{\endemptybar \vspace*{-1em} \endframed}

\newenvironment{mypar}[1]
  {\vspace{-.135in}
   \begin{paragraph}{#1}}%
  {\end{paragraph}}

\newenvironment{notate}[1]
  {\begin{nota}[{\bf {\em #1}}]}%
  {\end{nota}}

%%% Integrating commands from various papers + compatibility stuff
\newcommand{\Zizek}{\v{Z}i\v{z}ek}
\def\subsubsubsection#1{\paragraph{#1}}
%%% end of integrating commands from various papers

%% some additional TOC configuration options
\makeatletter
  \renewcommand\ext@table{lof}
\makeatother
\renewcommand*{\cfttablename}{Tab.\space}
\renewcommand*{\cftfigurename}{{\bf Fig.}\space}
\renewcommand*{\listfigurename}{List of Figures and Tables} % NOT WORKING

% scshape to set these to red
\renewcommand*{\cftchapterfont}{\slshape}   
\renewcommand*{\cftsectionfont}{\slshape}

%%%
%%%  BEGIN DOCUMENT!
%%%
\doublespacing
\begin{document}

\renewcommand{\arraystretch}{1.5}
\begin{table}
\begin{center}
\raisebox{3in}{\small
\begin{tabular}{|p{.25\textwidth}|p{.25\textwidth}|p{.25\textwidth}|}
\hline 
% \multicolumn{1}{p{.25\textwidth}}{\textbf{Relevance}}
\begin{center}(PM)
\end{center}

Ultimately relevance depends on peer review, and irrelevant content
may be deleted.  Mechanisms to ensure that relevant content
\emph{will} be added could be improved. &
\begin{center}(WP)
\end{center}
 People contribute articles about what
they're interested in; apart from this, rules like WP:WEIGHT come into
play.& 
\begin{center}(DO)
\end{center}

Anyone can upload projects (for ``full projects'', one time approval
is needed), but getting changes into the core requires considerably
more vetting. \\
%\multicolumn{1}{p{.25\textwidth}}{\textbf{Quality}}
Quality control is handled with corrections and the ``orphaning'' mechanism
in case of nonresponsive authors.  Some articles are world-writeable,
as in the wiki model. & Automated tools for spam and vandalism detection
combined with a system of editorial oversight, in which Jimmy Wales
has last say. & In addition to bug reports and feature requests handled
through the issue tracker, modules can make use of an automated patch
testing system.\\
%\multicolumn{1}{p{.25\textwidth}}{\textbf{Scalability}}
Peer review is distributed.  Links are handled automatically.  Caching
is deployed where relevant; in particular, interlinking features are
kept up to date. & The database and other infrastructure is massively
scaled. There are many bots that help with small tasks. & In theory,
anyone can join.  Earl Miles, NYCCamp 2012 keynote: ``\emph{There are
  no insiders, except Dries; there are no outsiders, only resumes.}''
\\
% \multicolumn{1}{p{.25\textwidth}}{\textbf{Consistency}}
Although automatic links and corrections can help with consistency,
mainly PM relies on standards for proof and expository quality.
&NPOV is the key rule, which works together with templates and other
process tools to maintain community standards about style and
content. & The project issue queues are the place to go when one
module's changes breaks another's. The core of the project has
considerable oversight in these
matters.\\
% \multicolumn{1}{p{.25\textwidth}}{\textbf{Motivation}}
People are solving some of their learning, exposition, and social
needs on the site by writing and reviewing articles and posting in the
forums.& As of 2006, over 50\% of the site had been written by less
than 1\% of the users; these days, paid editing is somewhat
notorious.& Miles continued: ``\emph{To build a resume, find someone
  who needs help, and help them.  Find something that needs doing, do
  it.}'' \\ \hline
\end{tabular}
}
\end{center}
\caption{As typeset ``out of the box''}
\end{table}

\renewcommand{\arraystretch}{1.5}
\begin{table}
\begin{center}
\raisebox{3in}{\small
\begin{tabular}{|p{.25\textwidth}|p{.25\textwidth}|p{.25\textwidth}|}
\hline 
% \multicolumn{1}{p{.25\textwidth}}{\textbf{Relevance}}
\begin{center}(PM)
\end{center}
\nohyphens{
Ultimately relevance depends on peer review, and irrelevant content
may be deleted.  Mechanisms to ensure that relevant content
\emph{will} be added could be improved. } &
\begin{center}(WP)
\end{center}
\nohyphens{ People contribute articles about what
they're interested in; apart from this, rules like WP:WEIGHT come into
play.}& 
\begin{center}(DO)
\end{center}
\nohyphens{
Anyone can upload projects (for ``full projects'', one time approval
is needed), but getting changes into the core requires considerably
more vetting.} \\
%\multicolumn{1}{p{.25\textwidth}}{\textbf{Quality}}
\nohyphens{Quality control is handled through corrections and the ``orphaning'' mechanism
in case of nonresponsive authors.  Some articles are world-writeable,
as in the wiki model.} &\nohyphens{ Automated tools for spam and vandalism detection
combined with a system of editorial oversight, in which Jimmy Wales
has last say.} &\nohyphens{ In addition to bug reports and feature requests handled
through the issue tracker, modules can make use of an automated patch
testing system.}\\
%\multicolumn{1}{p{.25\textwidth}}{\textbf{Scalability}}
\nohyphens{Peer review is distributed.  Links are handled automatically.  Caching
is deployed where relevant; in particular, interlinking features are
kept up to date. }& \nohyphens{The database and other infrastructure is massively
scaled. There are many bots that help with small tasks. }& \nohyphens{In theory,
anyone can join.  Earl Miles, NYCCamp 2012 keynote: ``\emph{There are
  no insiders, except Dries; there are no outsiders, only resumes.}''}
\\
% \multicolumn{1}{p{.25\textwidth}}{\textbf{Consistency}}
\nohyphens{Although automatic links and corrections can help with consistency,
mainly PM relies on standards for proof and expository quality.}
&\nohyphens{NPOV is the key rule, which works together with templates and other
process tools to maintain community standards about style and
content.} & \nohyphens{The project issue queues are the place to go when one
module's changes breaks another's. The core of the project has
considerable oversight in these
matters.}\\
% \multicolumn{1}{p{.25\textwidth}}{\textbf{Motivation}}
\nohyphens{People are solving some of their learning, exposition, and social
needs on the site by writing and reviewing articles and posting in the
forums.}& \nohyphens{As of 2006, over 50\% of the site had been written by less
than 1\% of the users; these days, paid editing is somewhat
notorious.}& \nohyphens{Miles continued: ``\emph{To build a resume, find someone
  who needs help, and help them.  Find something that needs doing, do
  it.}}'' \\ \hline
\end{tabular}
}
\end{center}
\caption{Attempt to control hypenation by using $\backslash$nohyphens\{...\}}
\end{table}

\end{document}
